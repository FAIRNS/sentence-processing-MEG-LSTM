\section{General Discussion}

\begin{itemize}
    \item briefly reiterate on the main findings 
    \item Humans-NLM comparison 
    \begin{itemize}
        \item Points of similarity: good performance on successive dependencies but difficulties on nested constructions; Subject-congruence effects on the main and embedded verbs; higher error rate on the embedded dependency (prediction 1). 
        \item Point of differences: size of effects. huge difference on Long-Nested - NLMs are below chance level. Prediction 2 was not confirmed. different short-range/compensation mechanisms.
    \end{itemize}
    
    \item comprehension vs. production.
    
    \item grammatical vs ungrammatical: in comprehension, effects are observed on the ungrammatical sentences. 
      
    \item In light of other models -
    \begin{itemize}
        \item ACT-R:
        \begin{itemize}    
             \item Can it account for the data? error-rates arise from interference. This is consistent for example with the observed subject-congruence effect. 
             \item For the NLM, both P*P*S and S*S*P are easy, despite the nested attractor and that the LR mechanism is taken up. Since the LR units output the number they encode throughout the dependency, their activity overcomes that of the attractor, which explains the good performance on these conditions. In general, interference dynamics in the NLM are implemented via competition between activity of units encoding for different numbers - e.g., LR of the main and SR for the embedded. 
             \item Consequently, can NLM seen as an implementation of higher-level, symbolic-based, mode such as the ACR-T?
        \end{itemize}
        
        \item Feature percolation models: cannot account for the incongruent effect on the embedded verbs. The attractor is higher on the tree and thus need to percolate `downwards'.
    \end{itemize}   

    \item Limitations: how to explain processing of cross dependencies, or a clause with sentential complement embedded inside an objrel (e.g, Gibson 98).
    \item Merits of the NLM model compared to other models.    
    
    
    \item Future directions. 
    
\end{itemize}