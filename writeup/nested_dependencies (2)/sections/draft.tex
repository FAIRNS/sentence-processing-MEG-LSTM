In these studies, participants made more errors when the head noun was singular and the attractor plural. Specifically, in comprehension, \citet{wagers2009agreement} showed that processing facilitation of an embedded dependency in ungrammatical sentences occurred when the attractor (the main subject in this case) was plural.


% \subsection{Number Agreement in Humans}
% Agreement in the psycholinguistic literature as a window into syntactic processing - a way to test theories regarding the evolving representations of syntactic structures in the mind of the subject. 

% Understanding the neural basis of high cognitive functions, such as those involved in language, requires a

Two of these units were shown to encode grammatical number (one for singular the other for plural) and robustly carry it across long-range dependencies. The activity of the third unit was shown to follow the structure of the long-range dependency and to convey this information to the two other `number' units via strong synaptic connections. The sparsity of this 

% Already one level of nesting, such as in object-extracted relative clauses "The boy that the sisters watch smiles", is known to be relatively hard to process for humans \citep[e.g., ][]{traxler2002processing}. In such constructions, an inner subject-verb dependency (`sisters'-`watch') is embedded inside a main one (`boy'-`smiles'), and the two should each agree on grammatical number - singular for the main and plural for the embedded dependency. One aspect of the difficulty of such nested constructions arises from the need to process two subject-verb dependencies that are active at once and that might also carry opposite grammatical numbers, as in the current example. 

Generally, the underlying syntactic representations during incremental parsing, and points of processing difficulties, can be probed in experiments by using number agreement as an index. Agreement-error pattern on various syntactic constructions provide rich empirical data to test models for online syntactic processing. Experiments on subject-verb dependencies have indeed led to the development of various hypotheses about online syntactic representations and processing, such as the `linear distance hypothesis' \citep{}, `clause packing hypothesis' \citep{Bock:Miller:1991}, `hierarchical distance hypothesis' \citep{franck2002subject, franck2006agreement} and the involvement of memory-retrieval processes in sentence processing \citep{lewis2005activation, wagers2009agreement, lago2015agreement}.

In another line of research, following recent advances in Natural Language Processing, number agreement has been also studied in Neural Language Models (NLMs) \citep{Linzen:etal:2016}. NLMs are neural-networks based models that are typically trained to predict the next word given a context on a large natural-language corpus \citep{Elman:1991}. NLMs were shown to develop near-human performance on various linguistic tasks \citep{}, which makes them compelling `hypothesis generators' in the study of the cognitive and neural basis of language processing in humans. Treating NLMs as psycholinguistic subjects, most current research in the field focuses on understanding the behaviour of NLMs with respect to various grammatical phenomena, with some work also showing correlations between internal states of the model and said phenomena. Number agreement is an area where some steps have been taken towards mechanistic understanding. Specifically, in a recent study, we showed that NLMs trained on a large corpus of English developed a number-propagation mechanism for long-range dependencies \citep{lakretz2019emergence}, which was found to comprise an exceptionally small number of units in the network, which were dedicated to carry number across long-range dependencies. Implications of this sparsity of the mechanism on number-agreement processing is the core motivation of the current study.