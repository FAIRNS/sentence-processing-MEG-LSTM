
\section{Related literature}

Previous studies on understanding RNNs can be roughly characterized in two classes: Studies that focus on visualising or interpreting the hidden state activations of such networks; and work that focuses on the behavior or performance of a model by studying the learnability of particular datasets or phenomena.
We briefly discuss the most important studies concerning RNNs applied for natural language processing.
For the second class, we specifically focus on behavioral studies concerning \textit{subject-verb} agreement, a topic particularly relevant for the current paper.

\paragraph{Interpreting internal states}
The first attempts to understand high-dimensional the internal representations of trained RNNs concerned visualisation based methods \cite{Karpathy:etal:2016,tang2017memory,li2016visualizing,Radford:etal:2017}.
Most related to this study, \newcite{Radford:etal:2017} found a ``sentiment'' grandmother cell and \newcite{Kementchedjhieva:Lopez:2018} found a character-level RNN to track morpheme boundaries in a single cell.
Other approaches use `probes' or `diagnostic classifiers': meta-models to extract information from the hidden states of an RNN \cite{Adi:etal:2017,Hupkes:etal:2017,alain2017understanding}.
This approach is not limited to finding properties that are encoded in a localist fashion and has yielded interesting linguistic results.
E.g., \newcite{gelderloos2016phonemes} find evidence that a multilayer RNN trained in a visually grounded learning paradigm learns to represent linguistic information in a hierarchy, encoding form in the lower and meaning in the higher layers.
\dieuwke{This is of course not very extensive, but maybe this is not really the place to put a too elaborate analysis?}

\subsection{Subject-verb agreement in English}
- Subject-verb agreement from psycholinguistics (e.g., Miller and Bock, Franck and Rizzi)
- SV agreemnt in LSTMs \cite{Linzen:etal:2016,Bernardy:Lappin:2017,Gulordava:etal:2018,Kuncoro:etal:2018a,Kuncoro:etal:2018b,Linzen:Leonard:2018}
- Relate to Nelson et. al 2017 PNAS, an intracranial study that identifies electrodes whose high-gamma activity correlates with syntactic tree-depth (numnber of open nodes)
