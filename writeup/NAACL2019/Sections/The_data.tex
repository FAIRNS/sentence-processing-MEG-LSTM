\section{Data}
% \textcolor{gray}{\lipsum[1]}

\begin{table}[tb]
  \centering
  \begin{footnotesize}
  \begin{tabular}{l@{\hskip1pt}l}
    \B Simple & the boy greets the guy\\
    \B Adv & the boy probably greets the guy\\
    \B 2Adv & the boy most probably greets the guy\\
    \B CoAdv &  the boy openly and deliberately greets the guy\\
    \B NamePP & the boy near Pat greets the guy\\
    \B NounPP & the boy near the car greets the guy\\
    \B SubjRel & the boy that avoids the girl greets the guy\\
    \B ObjRel  & the boy that the girl avoids greets the guy \\
    \B ObjRel0 &  the boy the girl avoids greets the guy\\
  \end{tabular}
  \end{footnotesize}
  \caption{Number agreement data-sets illustrated by representative
    singular sentences.}
  \label{tab:data-sets}
\end{table}

We generated data-sets with fixed syntactic structures and varied
lexical material that probe subject-verb number agreement in
increasingly challenging conditions. The different structures are
illustrated in Table \ref{tab:data-sets} by examples where all forms
are in the singular. Distinct sentences were randomly generated by
selecting (different) words from pools of 20 subject/object nouns, 15
verbs, 10 adverbs, 5 prepositions, 10 proper nouns and 10 location
nouns. The items were selected so that their combination would not
lead to semantic anomalies. We generated singular and plural
versions of each sentence. Moreover, where other nouns occur between
subject and main verb, we also systematically varied their number. For example, the
NounPP sentence in the table illustrates the SS (singular-singular)
condition. The corresponding sentences in the other conditions are:
"the boy near the cars greets the guy'' (SP), ``the boys near the car
greet the guy'' (PS), and ``the boys near the cars greet the guy''
(PP). In all cases, ungrammatical versions with (main-clause)
subject-verb number mismatches were also generated. For the NounPP
example in the table, the mismatched counterpart is ``the boy near
the car \textbf{greet} the guy''. Each correct/mismatched pair was
presented to the pre-trained network, and we computed its accuracy by
counting as hits all cases where it assigned a higher likelihood to
the correct version. Note that 2Adv features the same subject-verb
distance as NamePP, but without gender-carrying words acting as
possible distractors in the middle. Similarly, CoAdv can serve as a
distractor-free control for NounPP.

Data-set sizes ranged from 600 (Simple) to 18k sentences (relatives),
based on the possibilities for variation allowed by the combinatorics
given each structure.\footnote{We uploaded the data-set generation
  script as supplementary material.}

Finally, we also used the naturalistic, corpus-derived agreement test set of \newcite{Linzen:etal:2016}, in the version made available by \newcite{Gulordava:etal:2018}.

% \subsection{Synthetic data}

% \subsubsection{Stimuli for the number-agreement task (NA-task)}
% This section describes the generation process of synthetic sentence stimuli for the number-agreement task (NA-task). 
% These sets of stimuli were used to evaluate the performance of both full- and ablated models on the NA-task. 

% Each NA-task contained sentences with a fixed syntactic structure, such as ``Det Noun Adv Verb'' or ``Det Noun P Det Noun Verb'', and each task was composed of several \textit{conditions} depending on the possible assignments of grammatical number to the nonu(s) in the sentence. 
% For example, one NA-task contained sentences of the form: ``Det Noun Verb'', and had two conditions corresponding to the two possible number values of its noun. 
% So the first conditiond such as ``The boy runs'', and the ablated network was tested on predicting the correct verb form. 
% This task had two conditions, corresponding to the two possible assignments of grammatical number (singular or plural) to the main noun. 
% Another NA-task contained sentences of the form: ``Det Noun-1 P Det Noun-1 Verb'', such as ``The boy behind the girls jumps''. 
% This task had four conditions, corresponding to the four possible assignments of grammatical number to noun-1 and noun-2.

% The network was evaluated on predicting the correct verb form (singular or plural).

% \subsection{Corpus data}

% \textcolor{gray}{\lipsum[1]}
