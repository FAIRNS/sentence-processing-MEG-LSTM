\begin{abstract}
  Recent work has shown that LSTMs trained on the generic language
  modeling objective capture syntax-sensitive generalizations such as
  long-distance agreement. We have however no mechanistic
  understanding of how they accomplish this remarkable feature, and
  some have conjectured it depends on heuristics that do not truly
  take hierarchical structure into account. We present here a detailed
  study of the inner mechanics of gender tracking in LSTMs at the
  single neuron level. We discover that gender information is managed
  by very few ``grandmother cells'' in a localist
  fashion. Importantly, the behaviour of the gender cells is partially
  controlled by other units that are independently shown to track the
  syntactic structure of sentences. We conclude that LSTMs are, to
  some extent, implementing genuinely syntactic processing mechanisms,
  paving the way to a more general understanding of grammatical
  encoding in LSTMs.
  \end{abstract}
