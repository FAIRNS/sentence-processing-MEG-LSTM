
\section{The models}

The recurrent models that are the focus of our studies are long short term memory models \cite[LSTMs]{Hochreiter:Schmidhuber:1997}.
To inspect the internal dynamics of these models, we use additional linear models to extract information from the hidden state of trained language models.
In this section, we describe the language model we study (\ref{ssec:lstm_lm}) as well as the meta models we use to analyse the hidden states of this language model (\ref{ssec:dc}).

\subsection{LSTM language model}\label{ssec:lstm_lm}

\paragraph{Model description} We consider the pretrained LSTM LM that was made available by \cite{Gulordava:etal:2018}.\footnote{We report our results only for this model, but acertained the robustness of these results by repeating the tests for models trained with different seeds and drop-out values}
This model has two LSTM layers with 650 dimensions, an output layer with vocabulary size 50000 and an embedding layer of 650 dimensions.
In our experiments, we investigate the different components of both layers of the LSTM, given by the equations below:
\begin{align}
    h_t & = o_t\circ \tanh(c_t)\\ 
     c_t & = f_t\circ c_{t-1} + i_t\circ \widetilde{c}_t\\
     \widetilde{c}_t & = \tanh(W_cx_t + U_ch_{t-1} + b_c)\\
     f_t & = \sigma(W_fx_t + U_fh_{t-1} + b_f) \\
     i_t & = \sigma(W_ix_t + U_ih_{t-1} + b_i) \\
     o_t & = \sigma(W_ox_t + U_oh_{t-1} + b_o)
\end{align}

We will refer to the n$^{th}$ unit in the first and second layer with the terms \unit{1}{n} and \unit{2}{n}, respectively.

\paragraph{Model evaluation on NA-tasks}
Model performance was evaluated on each NA-task and conditions by computing its accuracy in predicting the correct form of the main verb. Specifically, for each sentence, the model was presented with all words up to the one immediately preceding the main verb. If the model assigned higher likelihood to the correct form of the verb than to its counterpart with the wrong number then the trial was marked as a hit. Model accuracy was then computed by counting the fraction of hits across the entire set of sentences in the condition, and is denoted in what follows as ACC_{NA-task, condition}.


\subsection{Regression model}\label{ssec:dc}

To understand which information is encoded in the hidden states and gates of the LSTM language model, we use a technique that is commonly used in neuroscience (ref?) but has recently also gained traction in the field of computational linguistics: training an additional model to predict information from these hidden states and gates \cite{Adi:etal:2017,Hupkes:etal:2017}.  

In our case, we aim to assess if the internal states of our language model incode information about the \textit{syntactic depth} of the sentence it is processing.
The setup is as follows.

- Explain that the model was used to predict the depth of the syntactic tree from network activity
\paragraph{Model description}
- Describe the Features: network activity (hidden/cell activity)
- Describe the label: Tree depth (\ref{ssec:n_opennodes})
- Describe the model: A Ridge/LASSO model.
\Describe{Model training and evalution}
- Model training: nested 5-fold CV procedure. Train/val/test. Optimal regularization size was estimated from the validation set (report optimal lamda - figures in FAIRNS.pdf on slack)
- model evaluation: R-squared on test set. Report resulting values (text+figures in FAIRNS.pdf). 


