\section{General Discussion}
% Re-iteration on the motivations and goals
We investigated how nested long-range number agreements are processed in Neural Language Models and humans. Previous inquires into NLMs found that during training, NLMs develop a sparse neural mechanism that specializes in the processing of long-range number agreements. These findings predict processing difficulties for NLMs on embedded long-range dependencies of nested constructions, since the sparsity of the agreement mechanism might not allow the processing of two dependencies that are active at once.

% Summary of the results on a single dependency in NLMs
\subsection{A Sparse Agreement Mechanism Consistently Emerges in NLMs Across Languages and Grammatical Features}
Using agreement tasks with a single subject-verb dependency, we first replicated previous findings from an English NLM and extended them to NLMs trained on Italian. We found that a similar sparse mechanism for number agreement emerged in Italian NLMs during training. Furthermore, we identified a similar mechanism for gender agreement, showing that a sparse agreement mechanism therefore consistently emerges in NLMs across languages and grammatical features. 

(discuss possible explanations for why this sparsity occurs so often)

% Summary of the results on two dependencies in NLMs
\subsection{Processing of Nested Dependencies in Humans and NLMs}
In the processing of two successive and nested constructions, we found considerable similarity between agreement-error patterns of NLMs and humans, however, with several important points of discrepancy:
\subsubsection{Similarities}
\begin{itemize}
    \item \textbf{Relatively low error-rate on successive dependencies}: humans and the NLM made a relatively small number of agreement errors on the embedded verb of successive dependencies. This is in accordance with the sequential processing observed in the dynamics of the NLM and human relatively good performance on right-branching constructions. 
    \item \textbf{Subject-congruence effect in nested constructions}: for all verbs, when the attractor was plural, both humans and NLMs made significantly more errors when the main and embedded subjects had opposite grammatical numbers. Moreover, in some cases, the size of subject-congruence effect was greater than that of an incongruent nested attractor. For example, for the embedded verb of Long-Nested, humans and the NLM made more errors due to incongruent subjects (effect-size=) compared to an intervening nested attractor (effect-size=; interaction). In the NLM, the effect of incongruent subjects in this case arises from interference dynamics. Throughout the processing of the main dependency, the long-range number unit output the value of the encoded grammatical number. In the case of incongruent subjects, this information interferes with information encoded in the short-range units that carry the number of the embedded subject. In contrast, in the case of congruent subjects, since the number of both subjects is identical, the activation of the long-range mechanism support the embedded dependency, also in the case of a nested incongruent attractor.
    \item \textbf{Embedded compared to main verbs of nested dependencies are more error prone}:
    
\end{itemize}

\subsubsection{Differences}
    \begin{itemize}
        \item effect sizes. 
	\item NLMs are below chance level. 
        \item Prediction 2 was not confirmed (different short-range/compensation mechanisms).
        \item markedness effect is absent in NLMs. 
    \end{itemize}
    
% The agreement mechanism of the NLM was found to process successive dependencies sequentially, achieving high performance on Short- and Long-Successive. In nested constructions, the NLM showed subject-congruence effects on all verbs, making more agreement errors when the main and embedded subject differ by their grammatical number. The NLM successfully handled the main dependency, in both Short- and Long-Nested, achieving significantly below chance-level performance also in the case of incongruent subjects. However, the model showed increased difficulty in processing the embedded dependencies in both Short- and Long-Nested, reaching below chance-level performance on the embedded dependency of Long-Nested in the case of incongruent subjects. These findings are in accordance with the prediction based on the sparsity property of the agreement mechanism discussed above (see also, Prediction 1 and 2 in Section \ref{}). 

% Summary of the results in humans
% Humans 
% Similarly to NLMs, humans showed high performance in both Short- and Long-Successive, with a higher baseline of errors across conditions, possibly due to attention lapses, which are not modelled in NLMs. These results are in accordance with the relatively high performance of humans in processing right-branching constructions. Similarly to NLMs, in nested constructions, humans showed subject-congruence effect across all verbs when the attractor was plural, making more errors in incongruent-subject cases. Furthermore, like NLMs, humans had relatively good performance on the main dependency, also when subjects were incongruent, and made significantly more errors on the embedded compared to the main dependency when subjects were incongruent (Prediction 1). However, although the error rate was higher, humans did not make significantly more errors on the embedded verb when the dependency was long- compared to short-range.

% differences

% similarities 

\begin{table}[h]
\tiny
\centering
\parbox{.4\linewidth}{
\centering
    \begin{tabular}{ |P{3cm}|P{1.3cm}|P{1.3cm}|  }
    \hline
    \rowcolor{cyan}
    \multicolumn{3}{|c|}{\textbf{Short-Successive}} \\
    \hline
    \rowcolor{cyan}
    \textbf{Effect} & \textbf{Humans} & \textbf{RNNs} \\
    \Xhline{3\arrayrulewidth}
    \rowcolor{green}
    \textbf{Subject Congruence (Embedded verb)} & X & X \\
    \hline
    
    \end{tabular}
}
\hfill
\parbox{.4\linewidth}{
\centering
    \begin{tabular}{ |P{3cm}|P{1.3cm}|P{1.3cm}|  }
    \hline
    \rowcolor{cyan}
    \multicolumn{3}{|c|}{\textbf{Long-Successive}} \\
    \hline
    \rowcolor{cyan}
    \textbf{Effect} & \textbf{Humans} & \textbf{RNNs} \\
    \Xhline{3\arrayrulewidth}
    \rowcolor{green}
    \textbf{Subject Congruence (Embedded verb)} & X & X \\
    \hline
    
    \end{tabular}
}

\vspace{25pt}

\parbox{.4\linewidth}{
\centering
    \begin{tabular}{ |P{3cm}|P{1.3cm}|P{1.3cm}|  }
    \hline
    \rowcolor{cyan}
    \multicolumn{3}{|c|}{\textbf{Short-Nested}} \\
    \hline
    \rowcolor{cyan}
    \textbf{Effect} & \textbf{Humans} & \textbf{RNNs} \\
    \Xhline{3\arrayrulewidth}
    \rowcolor{magenta}
    \textbf{Subject Congruence (Main verb)} & X &  V\\
    \Xhline{3\arrayrulewidth}
    \rowcolor{green}
    \textbf{Subject Congruence (Embedded verb)} & V &  V\\
    \hline
    \rowcolor{green}
    \textbf{Interaction: Subject-Congruence Verb-Position} & V &  V\\
    \hline
    
    \end{tabular}
}
\hfill
\parbox{.4\linewidth}{
\centering
    \begin{tabular}{ |P{3cm}|P{1.3cm}|P{1.3cm}|  }
    \hline
    \rowcolor{cyan}
    \multicolumn{3}{|c|}{\textbf{Long-Nested}} \\
    \hline
    \rowcolor{cyan}
    \textbf{Effect} & \textbf{Humans} & \textbf{RNNs} \\
    \Xhline{3\arrayrulewidth}
    \rowcolor{magenta}
    \textbf{Subject Congruence (Main verb)} & X &  V\\
    \Xhline{3\arrayrulewidth}
    \rowcolor{green}
    \textbf{Subject Congruence (Embedded verb)} & V &  V\\
    \hline
    \rowcolor{green}
    \textbf{Interaction: Subject-Congruence Verb-Position} & V &  V\\
    \hline
    
    \end{tabular}
}
\caption{A summary of all effects  humans and RNNs.}
\label{tbl:comparison}
\end{table}


\subsection{Psycholinguistic Theories for Agreement and Sentence Processing}
Several theories have been suggested in the psycholinguistic literature for grammatical agreement processing in humans, based on agreement-error data. We now discuss how these theories may account for the data collected in this study, which may provide a complementary high-level explanations to that suggested by the neural model. 



\subsubsection{Feature Percolation Theories}
% Some background (possibly unnecessary if dicussion gets too long)
Early psycholinguistic theories suggested that the proximity between an intervening noun and a verb determines the probability of making an agreement error \citep{quirk1972grammar}. This linear-distance hypothesis was later rejected by empirical findings showing that error rates across a prepositional phrase are higher compared to those across a relative clause, although in the former case, the subject was closer to the verb and the syntactic complexity of the preamble was smaller \citep{bock1992regulating}. Following these findings, Bock et. al suggested the clause-packaging hypothesis, stating that an attractor within the same clause would generate more interference than one in another structural unit. 

More recent studies suggested that agreement errors depend on the syntactic distance between the attractor and the point of possible interference with subject-verb agreement on the syntactic tree  \citep{vigliocco1995constructing, vigliocco1999sex, franck2002subject}. According to this view, the grammatical feature of the attractor can `percolate’ up through the syntactic-tree during incremental processing. Such feature percolation can influence the grammatical agreement between the head noun and verb. Feature percolation is assumed to occur incrementally during sentence processing, therefore, the longer the distance to the subject-verb path, the lower the likelihood of interference is. The syntactic-distance hypothesis accounts for the reduced error rates in relative-clauses compared to prepositional phrases (PP), and for more recent evidence to which the clause-packaging hypothesis provided inadequate predictions \citep{franck2002subject}.

However, percolation theories have difficulties to account for agreement errors on embedded dependencies, such as in Short- and Long-Nested \citep{wagers2009agreement, lago2015agreement}. In these constructions, the attractor with respect to the embedded dependency is the main subject, which resides higher on the syntactic tree. Percolation theories traditionally assume only upward movement through the tree and therefore predict null subject-congruence effect. Moreover, the syntactic distance between the main and embedded subjects is much greater than that between the subject and attractor in simple constructions as across a PP. This predicts low error-rates than those reported for PP constructions, in contrast to our findings. Finally, since the main subject does not linearly intervene within the embedded dependency, its grammatical feature should be less activated relative to the embedded subject. This is again in contrast to our findings (and note that it is opposite to the prediction from the NLM, in which the main subject is robustly encoded and carried by the agreement mechanism).


\subsubsection{Sentence Processing in The Adaptive Control of Thought---Rational (ACT-R) framework}
\begin{itemize}
    \item ACT-R:
    \begin{itemize}    
         \item Can it account for the data? error-rates arise from interference. This is consistent for example with the observed subject-congruence effect. 
         \item For the NLM, both P*P*S and S*S*P are easy, despite the nested attractor and that the LR mechanism is taken up. Since the LR units output the number they encode throughout the dependency, their activity overcomes that of the attractor, which explains the good performance on these conditions. In general, interference dynamics in the NLM are implemented via competition between activity of units encoding for different numbers - e.g., LR of the main and SR for the embedded. 
         \item Consequently, can NLM seen as an implementation of higher-level, symbolic-based, mode such as the ACR-T?
    \end{itemize}

\begin{itemize}
    
    \item comprehension vs. production.
    
    \item grammatical vs ungrammatical: in comprehension, effects are observed on the ungrammatical sentences. 
      
        
        \item Feature percolation models: cannot account for the incongruent effect on the embedded verbs. The attractor is higher on the tree and thus need to percolate `downwards'.
    \end{itemize}   

    \item Limitations: how to explain processing of cross dependencies, or a clause with sentential complement embedded inside an objrel (e.g, Gibson 98).
    \item Merits of the NLM model compared to other models.    
    
    
    \item Future directions. 
    
\end{itemize}

