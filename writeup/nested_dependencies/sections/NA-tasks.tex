\subsection{Number-Agreement Tasks}
We construct six \emph{number-agreement (NA)} tasks: two for the ablation studies, with a single long-range dependency, and four NA-tasks for the nesting experiments, having two long-range dependencies (Figrure \ref{fig:design}). We give a short description of these tasks below. Table~\ref{tab:na-tasks-overview} provides an overview on all tasks. 

\begin{table}[h]
    \setlength\tabcolsep{3mm}
\small
\centering
\begin{tabular}{lll}
\multicolumn{3}{c}{\centering \textit{Agreement tasks for ablation experiments}}\\
\hline
\hline
\emph{NounPP-number} & \texttt{\textbf{NP$_a$} prep NP$_b$ \emph{V$_a$}} & \specialcell{Il \textbf{ragazzo} accanto alla \underline{donna} \textbf{conosce}\vspace{-1mm}\\({\scriptsize The \textbf{boy} next to the \underline{woman} \emph{knows}})} \\
\emph{NounPP-gender} & \texttt{\textbf{NP$_a$} prep NP$_b$ BE$_a$ \emph{ADJ$_a$}} & \specialcell{Il \textbf{ragazzo} accanto alla \underline{donna} \`{e} \textbf{basso}\vspace{-1mm}\\({\scriptsize The \textbf{boy} next to the \underline{woman} is \textbf{short-m}})}\\
\vspace{-2mm}\\
\multicolumn{3}{c}{\centering \textit{Number-agreement tasks for nesting experiments}}\\
\hline
\hline
\emph{Short-Successive} & \texttt{NP$_a$ V$_a$ che NP$_b$ V$_b$} & \specialcell{Il \textbf{figlio} \textbf{dice} che il \emph{ragazzo} \emph{ama}\vspace{-1mm}\\{\scriptsize The \textbf{son} \textbf{says} that the \emph{boy} \emph{loves}}} \\
\emph{Long-Successive} & \texttt{NP$_a$ V$_a$ che NP$_b$ P NP$_c$ V$_b$} & \specialcell{Il \textbf{figlio dice} che l'\emph{amico} accanto al \underline{ragazzo} \emph{conosce}\vspace{-1mm}\\{\scriptsize The \textbf{son says} that the \emph{friend} next to the \underline{boy} \emph{knows}}} \\
\emph{Short-Nested} & \texttt{NP$_a$ che NP$_b$ V$_b$ V$_a$ } & \specialcell{Il \textbf{figlio} che il \emph{ragazzo} \emph{osserva} \textbf{evita}\vspace{-1mm}\\{\scriptsize The \textbf{son} that the \emph{boy} \emph{observes} \textbf{avoids}}} \\
\emph{Long-Nested} & \texttt{NP$_a$ che NP$_b$ P NP$_c$ V$_b$ V$_a$} & \specialcell{Il \textbf{figlio} che la \emph{ragazza} accanto ai \underline{padri} \emph{ama} \textbf{evita}\vspace{-1mm}\\{\scriptsize The \textbf{son} that the \emph{girl} next to the \underline{fathers} \emph{loves} \textbf{avoids}}} \\
\end{tabular}
\caption{\textbf{Agreement tasks for the ablation and nesting experiments.}
The first column denotes the name of the task, the second shows the sentence templates, where \texttt{NP} is used as an abbreviation of \texttt{Det N}.
The indices $a$, $b$ mark the subject-verb dependencies in the templates. 
%For example, in \emph{Long-Nested}, there are three nouns and two verbs, the indices $a$ and $b$ indicate that the last verb \texttt{V$_a$} is syntactically dependent on the first noun phrase \texttt{NP$_a$}, whereas the penultimate verb \texttt{V$_b$} instead should match the features of the second noun phrase \texttt{NP$_b$}.
Note that for \emph{Long-} and \emph{Short-Nested}, we test performance on both the \emph{embedded} verb \texttt{V$_b$} and the \emph{main} verb \texttt{V$_a$}.
The last column contains an example of a sentence in the corresponding agreement task, along with its English translation.
Bold and italic face highlight the dependencies marked by the indices in the templates.
For each agreement task, we systematically vary the \emph{number} (or gender, in \emph{NounPP-gender}) of all nouns in the template, resulting in four different conditions (SS, SP, PS and PP) for the number-agreement tasks with two nouns (\emph{NounPP-number}, \emph{NounPP-gender}, \emph{Short-Successive} and \emph{Short-Nested}) and eight different conditions (SSS, SSP, SPS, SPP, PSS, PSP, PPS and PPP) for the number-agreement tasks with three nouns (\emph{Long-Successive} and \emph{Long-Nested}).
The examples shown are all SS and SSS conditions. For  \emph{NounPP-gender}, singular (S) and plural (P) are replaced by feminine and masculine.
\label{tab:na-tasks-overview}}
\end{table}

\paragraph{Tasks for the Ablation Experiments} For the ablation studies, we construct NA-tasks with a single long-range dependency across a prepositional phrase \citep{lakretz2019emergence}. To identify long-range number units, we use NounPP-number (Table~\ref{tab:na-tasks-overview}), in which the main subject and verb agree on grammatical number, and a second noun (attractor) can interfere during sentence processing. To identify long-range gender units, we use NounPP-gender, in which the main subject and a predicative adjective agree on gender, and an attractor having an opposite gender can interfere in the middle. 

\paragraph{Tasks for the Nesting Experiments} For the nesting experiments, we use four different NA-tasks. All tasks contain two subject-verb dependencies; they differ in terms of whether these dependencies are \emph{successive} or \emph{nested} and whether the embedded dependency is \emph{short-} or \emph{long-range}. Two subject-verb dependencies are \emph{successive} when the second noun occurs only after the first has been resolved. 
Such dependencies occur, for instance, in declarative sentences with a sentential complement, such as ``Il \textbf{ragazzo} \textbf{dice} che la \textit{donna conosce} il cane'' (``The \textbf{boy says} that the \emph{woman knows} the dog'').
In this example, the subject and verb are directly adjacent in both dependencies.
We thus call these dependencies \emph{short-range}, and the above sentence is an example of a sentence that could occur in the \emph{Short-Successive} NA-task. 
To create sentences that have a successive but \emph{long-range} subject-verb relationships, we increase the distance between the second noun and its corresponding verb by inserting a prepositional phrase in between them: ``Il \textbf{ragazzo} \textbf{dice} che la \textit{donna} accanto al \underline{figlio} \textit{conosce} il cane' '(``The \textbf{boy says} that the \emph{woman} next to the \underline{son} \emph{knows} the dog''). For nested dependencies, we consider sentences with object-relative clauses, such as ``Il \textbf{ragazzo} che la \emph{figlia} \emph{ama} \textbf{conosce} il contadino'' (``The \textbf{boy} who the \emph{daughter} \emph{loves} \textbf{knows} the farmer'').
As the nested dependency in this sentence is short range, we call the corresponding number-agreement task \emph{Short-Nested}.
For \emph{Long-Nested}, we again use prepositional phrases to increase the distance between the subject and verb of the nested dependency: ``Il \textbf{ragazzo} che la \emph{figlia} vicino alla \underline{donna} \emph{ama} \textbf{conosce} il contadino'' (``The \textbf{boy} who the \emph{daughter} near the \underline{woman} \emph{loves} \textbf{knows} the farmer'').

% \paragraph{NounPP} To establish the presence of long-range \emph{number} units, we create an Italian version of \citeauthor{lakretz2019emergence}'s \citeyear{lakretz2019emergence} NounPP task.
% This data set consists of 4000 sentences in which the subject is separated from the main verb by a prepositional phrase.
% I.e., an example of a sentence in this dataset, which we dub \emph{NounPP}, is ``il \textbf{ragazzo} accanto alla \underline{donna} \textbf{conosce} \ldots'' (the \textbf{boy} next to the \underline{woman} \textbf{knows}).
% We systematically vary the number of both the subject and the noun in the prepositional phrase (also referred to with the term \emph{attractor}), resulting in four different conditions: singular-singular (SS), singular-plural (SP), plural-singular (PS) and plural-plural (PP).
% For all of these conditions, we generate 1000 sentences by randomly sampling (object and subject) nouns, verbs and prepositions from a list of options, reported in Table \ref{}.\dnote{I just realise that this is only true for the NLM experiments, where would we like to report these numbers.}
% \dieuwke{Should we say something about ensuring that sentences are at least somewhat semantically acceptable, i.e. excluding sentences like ``Il figlio sotto il cane \ldots''}

% \paragraph{NounPP-gender}
% To investigate if the model has long-range \emph{gender} units, we adapt the NounPP task to end with a predicative adjective.
% For example: ``il \textbf{ragazzo} accanto alla \underline{donna} \`{e} \textbf{basso} (the \textbf{boy} next to the \underline{woman} is \textbf{short}).
% In this NA-task, which we call \emph{NounPP-gender}, we focus on the adjective, which in Italian should match both the number and the gender of the subject.
% We again consider four different conditions: feminine-feminine (FF), feminine-masculine (FM) masculine-feminine (MF) and masculine-masculine (MM), within which the number of the subject and attractor is ranomly varied.
% Similar to the NounPP task, we sample 1000 sentences per condition, using the word lists provided in Table~\ref{}.



% \dieuwke{May 2: I am a bit unclear here about how many sentences we generated accross which conditions (I think there are 4032?), I'll first ask Yair before finishing this.}
% An overview can be found in Table~\ref{tab:na-tasks-overview}.



