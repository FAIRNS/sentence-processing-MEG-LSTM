\subsection{NA-tasks}

For both our ablation and nestedness experiments \dnote{do we have a name for this? I know it is not ``nestedness-experiments'', but I needed a way to refer to them in both the text and the tables.} we two and four different \emph{number-agreement (NA)} tasks, respectively.
We give a short description of these tasks below, an overview can be found in Table~\ref{tab:na-tasks-overview}.
The complete lexicon we used to generate all stimuli can be found in the supplementary materials, Table~\ref{}.

\subsubsection{Ablation experiments}
For our ablation studies, which we conduct to replicate and extend the findings of \citet{lakretz2019emergence} to Italian, we use two different types of NA tasks.

\paragraph{NounPP}
First, to establish the presence of long-range \emph{number} units, we create an Italian version of \citeauthor{lakretz2019emergence}'s \citeyear{lakretz2019emergence} NounPP task.
This data set consists of 4000 sentences in which the subject is separated from the main verb by a prepositional phrase.
I.e., an example of a sentence in this dataset, which we dub \emph{nounPP}, is ``il \textbf{ragazzo} accanto alla \underline{donna} \textbf{conosce} \ldots'' (the \textbf{boy} next to the \underline{woman} \textbf{knows}).
We systematically vary the number of both the subject and the noun in the prepositional phrase (also referred to with the term \emph{attractor}), resulting in four different conditions: singular-singular (SS), singular-plural (SP), plural-singular (PS) and plural-plural (PP).
For all of these conditions, we generate 1000 sentences by randomly sampling (object and subject) nouns, verbs and prepositions from a list of options, reported in Table \ref{}.\dnote{I just realise that this is only true for the NLM experiments, where would we like to report these numbers.}
\dieuwke{Should we say something about ensuring that sentences are at least somewhat semantically acceptable, i.e. excluding sentences like ``Il figlio sotto il cane \ldots''}

\paragraph{NounPP-gender}
To investigate if the model has long-range \emph{gender} units, we adapt the NounPP task to end with a predicative adjective.
For example: ``il \textbf{ragazzo} accanto alla \underline{donna} \`{e} \textbf{basso} (the \textbf{boy} next to the \underline{woman} is \textbf{short}).
In this NA-task, which we call \emph{NounPP-gender}, we focus on the adjective, which in Italian should match both the number and the gender of the subject.
We again consider four different conditions: feminine-feminine (FF), feminine-masculine (FM) masculine-feminine (MF) and masculine-masculine (MM), within which the number of the subject and attractor is ranomly varied.
Similar to the NounPP task, we sample 1000 sentences per condition, using the word lists provided in Table~\ref{}.

\subsubsection{Nestedness experiments}

For our nestedness experiments, we use four different NA-tasks.
All tasks contain two subject-verb dependencies; they differ in terms of whether these dependencies are \emph{successive} or \emph{nested} and whether they are \emph{long-distance} or \emph{short-distance}.

Two subject-verb dependencies are \emph{successive} when the second noun occurs only after the first has been resolved. 
Such dependencies occur, for instance, in declarative sentences, such as ``Il \textbf{ragazzo} \textbf{dice} che la \textit{donna conosce} il cane'' (``The \textbf{boy says} that the \emph{woman knows} the dog'').
In this example, the subject and verb are directly adjacent in both dependencies.
We thus call these dependencies \emph{short-distance}, and the above sentence is an example of a sentence that could occur in our \emph{successive-short} NA-task. 
To create sentences that have a successive but \emph{long-distance} subject-verb relationships, we increase the distance between the second noun and its corresponding verb by inserting a prepositional phrase in between them: ``Il \textbf{ragazzo} \textbf{dice} che la \textit{donna} accanto al \underline{figlio} \textit{conosce} il cane' '(``The \textbf{boy says} that the \emph{woman} next to the \underline{son} \emph{knows} the dog'').

An \emph{nested} subject-verb dependencies instead occur in a relative clause. 
We consider sentences with object-relative clauses, such as ``Il \textbf{ragazzo} che la \emph{figlia} \emph{ama} \textbf{conosce} il contadino'' (``The \textbf{boy} who the \emph{daughter} \emph{loves} \textbf{knows} the farmer'').
As the nested dependency in this sentence is a short distance dependency, we call this type of sentences \emph{short-nested}.
To create \emph{long-nested} sentences, we again use prepositional phrases to increase the distance between the subject and verb of the nested dependency: ``Il \textbf{ragazzo} che la \emph{figlia} vicino alla \underline{donna} \emph{ama} \textbf{conosce} il contadino'' (``The \textbf{boy} who the \emph{daughter} near the \underline{woman} \emph{loves} \textbf{knows} the farmer'').

\dieuwke{May 2: I am a bit unclear here about how many sentences we generated accross which conditions (I think there are 4032?), I'll first ask Yair before finishing this.}
An overview can be found in Table~\ref{tab:na-tasks-overview}.

\begin{table}[h!]
    \setlength\tabcolsep{2mm}
\small
\centering
\begin{tabular}{lll}
\multicolumn{3}{c}{\centering \textit{Probing number/gender units}}\\
\hline
\hline
\emph{Nounpp} & \texttt{\textbf{NP$_a$} prep NP$_b$ \emph{V$_a$}} & \specialcell{Il \textbf{ragazzo} accanto alla \underline{donna} \textbf{conosce}\vspace{-3mm}\\({\scriptsize The \textbf{boy} next to the \underline{woman} \emph{knows}})} \\
\emph{Nounpp-gender} & \texttt{\textbf{NP$_a$} prep NP$_b$ BE$_a$ \emph{ADJ$_a$}} & \specialcell{Il \textbf{ragazzo} accanto alla \underline{donna} \`{e} \textbf{basso}\vspace{-3mm}\\({\scriptsize The \textbf{boy} next to the \underline{woman} is \textbf{short}})}\\
~\\
\multicolumn{3}{c}{\centering \textit{Nesting experiments}}\\
\hline
\hline
\emph{Short-Successive} & \texttt{NP$_a$ V$_a$ che NP$_b$ V$_b$} & \specialcell{Il \textbf{figlio} \textbf{dice} che il \emph{ragazzo} \emph{ama}\vspace{-3mm}\\{\scriptsize The \textbf{son} \textbf{says} that the \emph{boy} \emph{loves}}} \\
\emph{Long-Successive} & \texttt{NP$_a$ V$_a$ che NP$_b$ P NP$_c$ V$_b$} & \specialcell{Il \textbf{figlio dice} che l'\emph{amico} accanto al \underline{ragazzo} \emph{conosce}\vspace{-3mm}\\{\scriptsize The \textbf{son says} that the \emph{friend} next to the \underline{boy} \emph{knows}}} \\
\emph{Short-nested} & \texttt{NP$_a$ che NP$_b$ V$_b$ V$_a$ } & \specialcell{Il \textbf{figlio} che il \emph{ragazzo} \emph{osserva} \textbf{evita}\vspace{-3mm}\\{\scriptsize The \textbf{son} that the \emph{boy} \emph{observes} \textbf{avoids}}} \\
\emph{Long-nested} & \texttt{NP$_a$ che NP$_b$ P NP$_c$ V$_b$ V$_a$} & \specialcell{Il \textbf{figlio} che la \emph{ragazza} accanto ai \underline{padri} \emph{ama} \textbf{evita}\vspace{-3mm}\\{\scriptsize The \textbf{son} that the \emph{girl} next to the \underline{fathers} \emph{loves} \textbf{avoids}}} \\
\end{tabular}
\caption{\textbf{The NA-tasks we use for our ablation and nesting experiments.}
The first column denotes the name of the task.
The second column indicates the format of the sentences, where \texttt{NP} is used as an abbreviation of \texttt{Det N}.
The indices $a$, $b$ help identify the noun-verb relationships in the templates.
I.e. in the \emph{long-nested} condition condition, there are three nouns and two verbs, the indices $a$ and $b$ indicate that the last verb \texttt{V$_a$} is syntactically dependent on the first noun phrase \texttt{NP$_a$}, whereas the penultimate verb \texttt{V$_b$} instead should match the features of the second noun phrase \texttt{NP$_b$}.
Note that for the long- and short-nested conditions, we test both the \emph{inner} verb \texttt{V$_b$} and the \emph{outer} verb \texttt{V$_a$}.
The third and last column contains an example of a sentence in the NA-task, along with its English translation.
Bold and italic face show the relations indicated by the indices in the templates.
For every NA-task, we systematically vary the \emph{number} (and in the gender-case also gender) of all nouns in the template, resulting in four different conditions (SS, SP, PS and PP) for the NA-tasks with two nouns (\emph{Nounpp}, \emph{Nounpp-gender}, \emph{Short-successive} and \emph{Short-nested}) and eight different conditions (SSS, SSP, SPS, SPP, PSS, PSP, PPS and PPP) for the NA-tasks with three nouns (\emph{Long-succesive} and \emph{Long-nested}).
The examples shown are all SS and SSS conditions.
\textbf{M: I agree that it might be a good idea to use NP, however this should also apply to the gender case \dieuwke{D: done}. Also, indexing is nice, but it should be used thoroughly \dieuwke{better like this?}. Also, it should also be shown in the examples. On the other hand, I am not sure of why only one noun/verb is capitalized/italicized for short- and long-nested, given that we test both main and embedded (perhaps, worth duplicating these cases?). \dieuwke{better like this?}}}\label{tab:na-tasks-overview}
\end{table}

