\section{Methods}

\subsection{Behavioral Experiment}
\subsubsection{Participants}
61 psychology students from the University of Milano-Bicocca (males = XX; Age = XX ± XX; Education = XX ± XX) took part in the experiment in exchange of course credits. Participants were Italian native speakers and were naïfs to the experiment purpose. The study was approved by the ethical committee of the Department of Psychology, and participants’ ethical treatment was in accordance with the principles stated in the Declaration of Helsinki.

\subsubsection{Stimuli}
Stimuli comprised i) acceptable sentences, in which both syntactic and semantic structure was correct; ii) number-violation trials, containing a violation in the subject-verb numerosity agreement; iii) filler sentences, comprising several syntactic and semantic violations. 
Acceptable sentences were created using a pool of XX nouns, combined with XX verbs to and XX complements.
Starting from this pool of sentences, number-violation and filler trials were created using MATLAB 2017b (The MathWorks, Natick, Massachusetts, United States).
Number-violation sentences were created by systematically changing V1 or V2 number agreement in objrel sentences and only V2 in embedded clause sentences. Given that all sentences contained third person (singular or plural) subjects, in this type of violation we replaced the singular form of one of the verbs with its plural one and viceversa (i.e. number violation il fratello che lo studente *accolgono ama i contadini, the brother that the student *welcome loves the farmers); 
Concerning fillers, syntactic and semantic violations were implemented. Syntactic violations consisted in i) replacing one of the verbs with the same verb in the wrong person, thus replacing third person verbs with first person ones and maintaining the singular or plural form (i.e. il fratello che lo studente *accolgo ama i contadini, the brother that the student *welcome loves the farmers); ii) replacing one of the two verbs with a noun from our pool (i.e. il fratello che lo studente *amica ama i contadini, the brother that the student *friend loves the farmers); iii) replacing one of the verbs with its infinitive form (i.e. il fratello che lo studente *accogliere ama i contadini, the brother that the student *to welcome loves the farmers). Moreover, to shift the attention from sentence’s verbs, we implemented fillers containing semantic violations, in which one of the animate nouns was substituted with an abstract (i.e. la *filosofia dice che la figlia ama la madre, the *philosophy says that the daughter loves the mother) or inanimate (i.e. la *matita dice che la figlia ama la madre, the *pencil says that the daughter loves the mother) noun. Considering that these trials used a different pool of nouns and were easily detectable as violations, half of them were replaced with acceptable fillers (i.e. abstract acceptable: il padre dice che la figlia ama la *filosofia, the father says that the daughter loves the *philosophy; inanimate acceptable: il padre dice che la *matita appartiene alla figlia, the father says that the *pencil belongs to the daughter) (see Section A – Supplementary materials for a more detailed description of stimuli generation).
Participants’ sentence lists comprised 540 sentences and were prepared before running the experiment, using a MATLAB script which randomly selected trials from a pool of sentences following specific constrains (see Section B - Supplementary materials for details). Each list comprised 180 acceptable sentences, 180 number violation trials and 180 fillers. Items’ order was randomized in each list and then automatically split in two parts (270 trials each), which were performed by participants in different days.

\subsubsection{Paradigm}
Participants took part in a two-session experiment, with each session lasting about forty-five minutes. The two sessions took place at the same time of the day at a maximum temporal distance of two weeks. After receiving information about the experimental procedure, participants were asked to sign the written informed consent. Participants were then sat in front of a 17” computer screen. They were informed that they would have been presented a list of sentences which could be acceptable or containing a syntactic or semantic violation. 
Stimuli were presented on a 17” computer screen in a light-grey, 30-point Courier New font on a dark background. Sentences were presented using a rapid serial visual presentation (RSVP) modality. Each trial stated with a fixation cross appearing at the center of the screen for 600 ms, then single words1 making up the sentence were presented, each one for 250 followed by 250 ms of black screen. At the end of each sentence, a blank screen was presented for 1500 ms, then a response panel appeared, reporting at the left and right of the screen two labels “correct” and “incorrect”, for maximum 1500 ms. A final screen, showing accuracy feedback was presented for 500 ms.
Participants were instructed to press as fast as possible the “M” key of the Italian keyboard when they detected a violation during sentence presentation. Sentences were presented entirely even when participants detected the violation before their end.  Then, in the response panel, participants were here asked to press the “X” key if the sentence was correct, or “M” whether the sentence was not correct. During the entire session, participants were asked to keep their left index over “X” ad their right index over “M” keys. After each trial participants received a feedback concerning their response: “Bravo!” (i.e. good!) in the case that sentence was correct, “Peccato..” (i.e. too bad) when it was incorrect.
At the beginning of each session, participants performed a training block comprising XX items. The training section included acceptable sentences, fillers and number-violation items and participants received a feedback on their accuracy after each trial. 
A participant did not perform the second session and his trials were not added to the analyses. 

\subsubsection{Data and Statistical Analyses}


\subsection{Language Model}

\subsubsection{Model Description}

The model architecture we use is an LSTM identical to the architecture presented by consists \citet{Gulordava:etal:2018}.
It consists of two LSTM layers with 650 units, an embedding layer of 650 units and a vocabulary with 50000 tokens.
The input and output embeddings of the model are not shared.

\subsubsection{Model Training} 
We train 20 language models on a corpus drawn from wikipedia data.
Contrary to common in language modelling, we train the models on separate sentences, rather than longer pieces of discourse.
We train 20 models, that differ in the order in which those sentences are presentedas well as the initialisation of their weights.
For all runs, we use a learning rate of 20, a batch size of 64 and a dropout rate of 0.2, the hyperparameters that \citet{Gulordava:etal:2018} reported to work best for this particular corpus.
As common practice for training language models, we do not use an optimiser, but instead use a \emph{plateau-based} learning scheme, in which we half the learning rate whenever the validation perplexity of the model reaches a plateau.

\subsubsection{Model Evaluation} Perplexity, Linzen behavioral tests.
We evaluate the resulting 20 models by considering their perplexity on a shared test set\footnote{\url{https://dl.fbaipublicfiles.com/colorless-green-rnns/training-data/Italian/test.txt}}, as well as their performance on the agreement tests provided by \citet{}.

\subsubsection{Ablation Experiments}
To identify units that play an important role in the encoding of number and gender, we run a series of ablation test.
In these ablation tests, we assess the impact of a \emph{single unit} on the models NA performance (\dieuwke{mention which test}) by setting its activation to zero and then recomputing the model's accuracy on the NA task. 
We conduct such ablation studies for all recurrent units in the network, resulting in 1300 ablation studies per model.

\subsubsection{Unit Activations}


