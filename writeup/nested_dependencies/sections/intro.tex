\section{Introduction}

\begin{itemize}
\item Competence vs performance - infinite recursion in linguistic vs. processing failure on shallow structures as studied in psycholinguistics. Here, we focus on 'performance', as constrained by actual cognitive/brain limited resources, in the ultimate goal of accounting for the rich phenomena found in behavioral and electrophysiological data.
\item Agreement in the psycholinguistic literature as a window into syntactic processing - a way to test theories regarding the evolving representations of syntactic structures in the mind of the subject. 
\item Neural Language Models as psycholinguistic subjects - revived interest in  NLMs as computational models of language processing. Ideally, NLMs would provide us with quantitative and predictive models for both behavior, coarse neural mechanisms and possibly language acquisition. 
\item Agreement in NLMs - current research in the area focuses on understanding the behaviour of NLMs wrt various grammatical phenomena; some work showing correlations between internal states and said phenomena.
\item Number agreement is an area where some steps have been taken
  towards mechanistic understanding:
  \begin{itemize}
  \item sparse feature-propagation mechanism controlled by distributed
    grammar network
  \end{itemize}
\item Our goal here is two-fold:
  \begin{itemize}
  \item First, we replicate and extend our earlier results on the emergence of
    this mechanism to a different language, as well as extending them to a different agreement phenomenon, confirming it is no fluke.
  \item Second, we use our understanding of how agreement is
    performed by NLMs to make a new prediction about a difficulty in agreement
    processing in humans (analogy to DiCarlo's work in vision?)
  \end{itemize}
\item Main result: the predicted pattern is confirmed; however, in other ways,
  human and NLM patterns differ, confirming that studying NLMs is
  productive, but the right perspective is not to treat them as
  full-fledged cognitive models, but rather as powerful computational
  systems that, when faced with challenges similar to those
  encountered by humans, might adopt partially analogous solutions
\end{itemize}
