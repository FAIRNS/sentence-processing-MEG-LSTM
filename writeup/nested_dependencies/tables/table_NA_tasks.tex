\begin{table}[h]
    \setlength\tabcolsep{3mm}
\small
\centering
\begin{tabular}{lll}
\multicolumn{3}{c}{\centering \textit{NA-tasks for ablation experiments}}\\
\hline
\hline
\emph{NounPP-number} & \texttt{\textbf{NP$_a$} prep NP$_b$ \emph{V$_a$}} & \specialcell{Il \textbf{ragazzo} accanto alla \underline{donna} \textbf{conosce}\vspace{-1mm}\\({\scriptsize The \textbf{boy} next to the \underline{woman} \emph{knows}})} \\
\emph{NounPP-gender} & \texttt{\textbf{NP$_a$} prep NP$_b$ BE$_a$ \emph{ADJ$_a$}} & \specialcell{Il \textbf{ragazzo} accanto alla \underline{donna} \`{e} \textbf{basso}\vspace{-1mm}\\({\scriptsize The \textbf{boy} next to the \underline{woman} is \textbf{short}})}\\
\vspace{-2mm}\\
\multicolumn{3}{c}{\centering \textit{NA-tasks for nesting experiments}}\\
\hline
\hline
\emph{Short-Successive} & \texttt{NP$_a$ V$_a$ che NP$_b$ V$_b$} & \specialcell{Il \textbf{figlio} \textbf{dice} che il \emph{ragazzo} \emph{ama}\vspace{-1mm}\\{\scriptsize The \textbf{son} \textbf{says} that the \emph{boy} \emph{loves}}} \\
\emph{Long-Successive} & \texttt{NP$_a$ V$_a$ che NP$_b$ P NP$_c$ V$_b$} & \specialcell{Il \textbf{figlio dice} che l'\emph{amico} accanto al \underline{ragazzo} \emph{conosce}\vspace{-1mm}\\{\scriptsize The \textbf{son says} that the \emph{friend} next to the \underline{boy} \emph{knows}}} \\
\emph{Short-Nested} & \texttt{NP$_a$ che NP$_b$ V$_b$ V$_a$ } & \specialcell{Il \textbf{figlio} che il \emph{ragazzo} \emph{osserva} \textbf{evita}\vspace{-1mm}\\{\scriptsize The \textbf{son} that the \emph{boy} \emph{observes} \textbf{avoids}}} \\
\emph{Long-Nested} & \texttt{NP$_a$ che NP$_b$ P NP$_c$ V$_b$ V$_a$} & \specialcell{Il \textbf{figlio} che la \emph{ragazza} accanto ai \underline{padri} \emph{ama} \textbf{evita}\vspace{-1mm}\\{\scriptsize The \textbf{son} that the \emph{girl} next to the \underline{fathers} \emph{loves} \textbf{avoids}}} \\
\end{tabular}
\caption{\textbf{Number-Agreement (NA) tasks for the ablation and nesting experiments.}
The first column denotes the name of the task, the second column the template of the sentences, where \texttt{NP} is used as an abbreviation of \texttt{Det N}.
The indices $a$, $b$ mark the subject-verb dependencies in the templates. 
%For example, in \emph{Long-Nested}, there are three nouns and two verbs, the indices $a$ and $b$ indicate that the last verb \texttt{V$_a$} is syntactically dependent on the first noun phrase \texttt{NP$_a$}, whereas the penultimate verb \texttt{V$_b$} instead should match the features of the second noun phrase \texttt{NP$_b$}.
Note that for Long- and Short-Nested, we test performance on both the \emph{embedded} verb \texttt{V$_b$} and the \emph{main} verb \texttt{V$_a$}.
The last column contains an example of a sentence in the corresponding NA-task, along with its English translation.
Bold and italic face highlight the dependencies indicated by the indices in the templates.
For each NA-task, we systematically vary the \emph{number} (or gender, in NounPP-gender) of all nouns in the template, resulting in four different conditions (SS, SP, PS and PP) for the NA-tasks with two nouns (\emph{NounPP-number}, \emph{NounPP-gender}, \emph{Short-Successive} and \emph{Short-Nested}) and eight different conditions (SSS, SSP, SPS, SPP, PSS, PSP, PPS and PPP) for the NA-tasks with three nouns (\emph{Long-Successive} and \emph{Long-Nested}).
The examples shown are all SS and SSS conditions.
\label{tab:na-tasks-overview}}
\end{table}